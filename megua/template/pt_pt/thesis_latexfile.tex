{# see MegBookWeb.py thesis function #}

\documentclass[a4paper]{book}



\usepackage{alltt}


\usepackage[layoutwidth=210mm,layoutheight=295mm,nomarginpar,nofoot,
textwidth=170mm,textheight=240mm,inner=25mm,verbose,outer=20mm,top=25mm,centering,
footskip=22pt]{geometry}



%Usado na escolha múltipla para desenhar um quadradinho
\newcommand{\quadra}{$\sqcup \!\!\!\!\sqcap$}
\def\questao{\item[] \quadra\hspace*{0.5cm}}


{# NOTE:  {{ '{#' }} means '{#' in the output #}


\newcommand{\reffonte}[2]{
\vfill\noindent Concretização do exercício~\textbf{{ '{#1}' }}~(ver código fonte na pág. \pageref{{ '{#2}' }}): \par
}


\newcommand{\cfonte}[2]{
\noindent \hrulefill~\textbf{{ '{#1}' }}~\hrulefill\label{{ '{#2}' }}
}


\begin{document}


\frontmatter

\tableofcontents


\mainmatter

\chapter{Introdução}

Aqui algum texto.

\chapter{Questões concretizadas}

\textbf{INSTRUÇÕES:}
\begin{itemize}
\item AS CONCRETIZAÇÕES ABAIXO PRECISAM DE AFINAÇÃO CASO A CASO POIS SÃO CONVERSÕES DO HTML PARA O LATEX. 
\item ALGUMAS ESTÃO PRONTAS E OUTRAS PRECISAM DE BASTANTE TRABALHO DE CONVERSÃO.
\item POR ISSO ESTÃO EM MODO VERBATIM. 
\begin{center}
\verb+\begin{verbatim}.....\end{verbatim}+
\end{center}
QUE DEVE SER REMOVIDO, EXERCÍCIO A EXERCÍCIO, COMPILANDO PRIMEIRO.
\end{itemize}



%%% Questões vindas do MEGUA daqui para baixo.

{{ungroupedquestions}}



\appendix

\chapter{Código Fonte}


\textbf{INSTRUÇÕES:}
\begin{itemize}
\item O MODO VERBATIM NÃO DEVE SER REMOVIDO.
\item SE DESEJAR QUEBRE ALGUMAS LINHAS PARA QUE O CONTEÚDO SEJA VISÍVEL NA FOLHA DE PAPEL.
\end{itemize}

{{sourcecode}}


\end{document}


