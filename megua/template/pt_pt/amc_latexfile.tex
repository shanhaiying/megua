{# see MegBookWeb.py amc functions #}

\documentclass[a4paper]{article}

\usepackage[utf8x]{inputenc}
\usepackage[T1]{fontenc}


%Como instalar o automultiplechoice package no Windows ?
\usepackage[box,completemulti]{automultiplechoice}

%Este ficheiro deve ser o jpedro a fornecer e 
%a ser colocado ao lado do ficheiro de exame/teste/tpc a compilar
\usepackage{amsfonts}
\usepackage{amcpt}


%Tirar comentário quando a versão for definitiva.
%Este comando escreve "rascunho" na folha A4.
%\AMCtext{draft}{}
%\AMCtext{none}{Nenhuma das respostas fornecidas é correta.}


\begin{document}


%%% Exemplo de utilização de questões agrupadas por tópicos
%%% As questões "\begin{question}" geradas estão no final deste ficheiro mas
%%% podem ser agrupadas como os exemplos seguintes.

\element{grupogeografia}{ %elemento do grupogeografia
  \begin{question}{Paris}
    Em que continente fica Paris?
    \begin{choices}
      \correctchoice{Europa}
      \wrongchoice{África}
      \wrongchoice{Ásia}
      \wrongchoice{Oceania}
    \end{choices}
  \end{question}
}

\element{grupogeografia}{ %elemento do grupogeografia
  \begin{question}{Cameroon}
    Qual é a capital dos Camarões?
    \begin{choices}
      \correctchoice{Yaoundé}
      \wrongchoice{Douala}
      \wrongchoice{Abou-Dabi}
    \end{choices}
  \end{question}
}

\element{grupohistoria}{ %elemento do grupohistoria
  \begin{question}{Bucaco}
    Em que ano ocorreu a batalha do Buçaco?
    \begin{choiceshoriz}
      \correctchoice{1810}
      \wrongchoice{1910}
      \wrongchoice{1710}
      \wrongchoice{1974}
    \end{choiceshoriz}
  \end{question}
}

\element{grupohistoria}{
  \begin{questionmult}{Bucaco2}
    O que pode ser afirmado sobre a batalha do Buçaco?
    \begin{choices}
      \correctchoice{Foi uma batalha aquando das invasões francesas}
      \correctchoice{As forças francesas entraram pelo eixo Ciudad Rodrigo -- Almeida -- Coimbra}
      \wrongchoice{No fim da batalha, a vitória mostrava-se nitidamente do lado francês}
    \end{choices}
  \end{questionmult}
}

%%% Questões vindas do MEGUA daqui para baixo.

{{ungroupedamcquestions}}


%%% copies

\onecopy{3}{   %%% especificar o número de réplicas (com questões e escolhas baralhadas).  


%%% Início do cabeçalho da prova

\noindent{\bf DISCIPLINA  \hfill Teste 1 de DISCIPLINA}

\vspace*{5mm}
\large\bf Número mecanográfico \hfill  
DATA
\vspace*{5mm}


{  % Número mecanográfico com bolinhas.
\setlength{\parindent}{0pt}\hspace*{\fill}\AMCcode{etu}{5}\hspace*{\fill}
\begin{minipage}[b]{9cm}
\begin{itemize}
\item $\longleftarrow{}$ Anote o seu número mecanográfico pintando cada retângulo apropriado. Se o seu número mecanográfico tiver menos de 5 algarismos use `0' no início.
\item Escreva o seu nome na caixa em baixo.
\end{itemize}
\vspace{3ex}
\hfill\namefield{\fbox{
\begin{minipage}{.95\linewidth}
Nome:\\
\vspace*{5mm} \\
%\dotfill
%\vspace*{1mm}
\end{minipage}
}}
%\hfill
%\vspace{5ex}
\end{minipage}
\hspace*{\fill}
}

%%% Fim do cabeçalho da prova


%%%Espaço para questões abertas
%%%O professor marca a pontuação antes da digitalização.

\begin{question}{open}
  Define \emph{democracia}.
  \AMCOpen{lines=10}{
        \wrongchoice[P0]{0}\scoring{0}
        \wrongchoice[P1]{1}\scoring{1}
        \correctchoice[P2]{2 (100\%)}\scoring{2}
  }
\end{question}


%%% Seleção de temas: cada tema tem várias questões.

\begin{center}
  \hrule\vspace{2mm}
  \bf\Large Geografia
  \vspace{1mm}\hrule
\end{center}

\shufflegroup{grupogeografia}
\insertgroup{grupogeografia}

\begin{center}
  \hrule\vspace{2mm}
  \bf\Large História
  \vspace{2mm}\hrule
\end{center}

\shufflegroup{grupohistoria}
\insertgroup{grupohistoria}

\begin{center}
  \hrule\vspace{2mm}
  \bf\Large Matemática
  \vspace{2mm}\hrule
\end{center}

% Comando opcional para formatar, por exemplo, em duas ou mais páginas A4.
% \clearpage 

\shufflegroup{grupomatematica}
\insertgroup{grupomatematica}


\clearpage %Obrigatório

}

\end{document}


%%% EXEMPLO de QUESTAO NAO AGRUPADA caso seja para criar à mão.
%
%\begin{question}{E51M25_NTopologicas_0001}
%O conjunto    
%$$D= \{ (x,y) \in \mathbb{R}^2 : 3 \, x^{2} + y^{2} < 1 \}$$
%\begin{choices}
%\correctchoice{ 
%é um conjunto aberto.
%}
%\wrongchoice{ 
%é um conjunto fechado.
%}
%\wrongchoice{ 
%é um conjunto que não é aberto nem fechado.
%}
%\wrongchoice{ 
%é um conjunto simultaneamente aberto e  fechado.
%}
%\end{choices}
%\end{question}
%
%
